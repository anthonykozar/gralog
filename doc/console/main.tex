\documentclass{article}

\usepackage[T1]{fontenc}

\usepackage[margin=1.0in]{geometry}
\usepackage{longtable}

\usepackage{amsmath}

\newcounter{example}
\newenvironment{example}[1][]{\refstepcounter{example}\par\medskip
   \noindent \textit{Example~\theexample. #1} \rmfamily}{\medskip}

\usepackage{bbm}
\usepackage{booktabs}

\newcommand{\fabs}[1]{\mid {#1} \mid}
\usepackage{hyperref}

\newcommand{\bet}[1]{\llbracket {#1} \rrbracket^{\beta} }
\usepackage{stmaryrd}
\usepackage{graphicx}
\usepackage{tikz}
\usepackage{wrapfig}
\usepackage{makecell}

% \usepackage{mathabx}
\usepackage{amssymb,forest}

\usepackage{float}
\usepackage{MnSymbol}
\newlength\q
\newlength\smallCol
\newlength\argsLen
\setlength\q{\dimexpr .5\textwidth -2\tabcolsep}
\setlength\smallCol{\dimexpr .15\textwidth}
\setlength\argsLen{\dimexpr .2\textwidth}


\newcommand{\lto}{\mathbin{\to}}
\usepackage{booktabs}
\usepackage{enumitem} 
\usepackage{array}% for extended column definitions
\usepackage{graphicx}
\usepackage{verbatim}
\usepackage{tabto}
\newcommand{\ov}[2]{\ensuremath{\overset{\cdot {#2} \cdot}{#1}}}
\newcommand{\imp}{\rightarrow}

\usepackage{xstring}
\usepackage[german]{babel}
\usepackage[utf8]{inputenc}

\usepackage{lipsum}
\usepackage{listings}
\usepackage{color}

\definecolor{dkgreen}{rgb}{0,0.6,0}
\definecolor{gray}{rgb}{0.5,0.5,0.5}
\definecolor{mauve}{rgb}{0.58,0,0.82}
\definecolor{palegreen}{rgb}{0.9,1,0.9}

\usepackage[most]{tcolorbox}
\tcbset{
    frame code={}
    center title,
    left=0pt,
    right=0pt,
    top=0pt,
    bottom=0pt,
    colback=palegreen,
    colframe=white,
    width=\dimexpr\textwidth\relax,
    enlarge left by=0mm,
    boxsep=5pt,
    arc=0pt,outer arc=0pt,
  }




\lstset{frame=tb,
  language=Java,
  aboveskip=3mm,
  belowskip=3mm,
  showstringspaces=false,
  columns=flexible,
  basicstyle={\small\ttfamily},
  numbers=none,
  numberstyle=\tiny\color{gray},
  keywordstyle=\color{blue},
  commentstyle=\color{dkgreen},
  stringstyle=\color{mauve},
  breaklines=true,
  breakatwhitespace=true,
  tabsize=3
}
\lstset{language=Python}

\title{Gralog External Programming Manual}

\author{Roman
  Rabinovich\\ \texttt{roman.rabinovich@tu-berlin.de}}

\date{August 2018}

\begin{document}

\maketitle


\begin{abstract}
Gralog is a visual tool for working with graphs, logics, games,
transition systems and other structures based on undirected and
directed graphs. It can create, load, save and edit graphs in
various formats.

The key focus of Gralog is simplicity of use and a short time of
learning how to use it.

A special property of Gralog is that it helps the developer to write
programmes for graphs in any language capable of working with
pipes. Gralog visualises the run of the programme and can keep track
of values of user defined variables.

The interaction between Gralog and the external programme is
performed by a simple, but powerful protocol. In the first version
we implemented a library for Python that simplifies the interaction
and abstracts away the use of pipes. This paper describes the
protocol and the library, the External Programming Module (EPM). This
includes documentation of methods and classes pertaining to the
EPM and code examples for how to use these.
\end{abstract}

\section{Documentation and Installation}

You can download Gralog from \url{www.gralog.org} (Change this)

Gralog is written in Java and does not need special installation. Just
run the jar file by

[code, including where the jar file is]

or run Gralog directly from the source code by calling

[code: ./gradlew].

The documentation can be found in directory \texttt{doc/}. The main
manual file is \texttt{gralog.pdf}. This part is additionally in the
file \texttt{external.pdf}. [TODO: make a general manual, include this
text there.]

\section{Working with the console}

The console is a tool to work with graphs by entering commands rather
than by clicking the mouse buttons. This is especially convenient if
you want to create large subgraphs with a regular pattern.

By default, the console is the frame below the main frame with the
graph. Here, in the bottom line, you can enter commands just like in a
terminal.

\subsection{Lists}

The objects you work with are \emph{lists:} lists of vertices and
lists of edges. We concentrate on vertex lists in this manual, edge
lists behave similarly. Where appropriate, we point out the
differences.

The lists are used to connect them later.


\begin{example}[Running example]
  For example, if one wants to create a perfect matching with $50$
  edges, one way to do that in Gralog is to create two disjoint lists
  of vertices and then connect them by drawing an edge from vertex $i$
  in the first list to vertex $i$ in the second list. Well, for a
  matching we would use a generator (see
  Section~\ref{sec:generators}), but what if we want to add edges such
  that already exiting $100$ vertices build a matching?
  \label{ex:running}
\end{example}


\subsection{Creating lists, (de)selecting elements}

All lists can be created out of the default list of all
vertices and the internal list of all currently selected vertices. The
latter can be made with the mouse in the obvious way or you can write
in the command line:
\begin{tcolorbox}
  select all vertices,
\end{tcolorbox}
\noindent
to select all vertices. By the way, in a similar way, all vertices are
deselected:

\begin{tcolorbox}
deselect all vertices.
\end{tcolorbox}

Now use the selection to create a new list. The corresponding command
is \texttt{filter}: one filters the elements from an existing list
according to certain criteria and saves them into another list. If the
other list already exists, the new elements are appended to it (even
if they are already contained in the target list). 

\begin{tcolorbox}
  filter all selected where no condition to VL1
\end{tcolorbox}

When creating list ids as \texttt{VL1}, keep in mind that the names
must be unique, in particular, it is impossible that a vertex list and
an edge list have the same name.

\subsection{Filter}

Now we created a new list \texttt{VL1} containing all vertices of the
graph. Let us now create a new list that contains only vertices with a
certain property, say those whose colour is blue. We can now use
\texttt{VL1} as a source of vertices.

\begin{tcolorbox}
  filter VL1 where fill color blue to VLblue
\end{tcolorbox}

Now the list \texttt{VLblue} contains all vertices with fill color
\texttt{blue}. Ne create a new list containing all vertices whose fill
color is red. Note in the following example that instead of the
keyword \texttt{where} one can write \texttt{such that} or
\texttt{st}. Note also that instead of \texttt{fill color} it is
possible to use just \texttt{fill}:

\begin{tcolorbox}
  filter VL1 st fill red to VLred
\end{tcolorbox}

The full specification of the filter command is as follows. Note that
Gralog does not distinguish between the upper and the lower cases.


\begin{tcolorbox}[colback=lightpink]
  FILTER <what> WHERE|ST|(SUCH THAT) <parameters> TO <list identifier>
  [ignored trash]
\end{tcolorbox}
 
\noindent where

\begin{tcolorbox}[colback=lightpink]
  \begin{tabular}[H]{rl}
    <what> := &ALL VERTICES | ALL EDGES\\
              &| [ALL ]? SELECTED VERTICES | [ALL ]? SELECTED EDGES\\
              & | <list id> \\[0.3cm]
    <parameters> := &<parameters> <parameters> | <boolean parameter>\\
              & | <fixed range parameter> <value>\\
              &| <numerical parameter> <cmp> <value> \\[0.3cm]
%                    & | FORMULA <integer expression(i)>\\
    <fixed range parameter> := & SHAPE | TYPE | EDGE TYPE | EDGETYPE \\[0.3cm]
    <numerical parameter> := & <integer parameter> | <float
                               parameter> | <string parameter>\\[0.3cm]
    <integer parameter> := & ID | DEGREE | INDEGREE |
                             OUTDEGREE\\[0.3cm]
    <float parameter> := & FILL | FILL COLOR | STROKE | STROKE
                       COLOR\\
                     &| THICKNESS | WIDTH | HEIGHT | SIZE |
                       WEIGHT\\[0.3cm]
    <string parameter> := &LABEL CONTAINS\\
    <boolean parameter> := &NO CONDITION | HAS SELFLOOP | DIRECTED | HAS
                             LABEL\\
              &| LABEL EMPTY | NO LABEL \\
                     &| BUTTERFLY // not implemented yet
  \end{tabular}
\end{tcolorbox}

In other words, we can write something like

\begin{tcolorbox}
  FILTER sourceList ST FILL RED STROKE BLUE SHAPE ELLIPSE DEGREE > 3 TO targetList
\end{tcolorbox}

Let us look at the grammar closer. In \texttt{<what>} there is no
difference between \texttt{ALL SELECTED VERTICES} and \texttt{SELECTED
VERTICES}, the result will be the same. The \texttt{<list id>} is the
name of a list we defined previously, it can be a list of vertices or
a list of edges.

The order of parameters in the parameter list is arbitrary. The list
of parameters cannot be empty. If you do not want any specifications,
choose \texttt{NO CONDITION}. In that case even if other parameters
are given, they are not taken into account and the whole source list
is given. This can be in particular used to duplicate every element in
a list:

\begin{tcolorbox}
  FILTER sourceList ST NO CONDITION TO newList

  FILTER sourceList ST NO CONDITION TO newList
\end{tcolorbox}

In this case \texttt{newList} will contain every element of
\texttt{sourceList} twice. By the way, being in the command line, one
can press $\uparrow$ (the arrow up key) to call previous commands.

Table~\ref{tab:parsAndVals} explains which values the parameters can have.

\begin{table}
  \centering
  \begin{tabular}{p{3cm}|p{5cm}|p{1.8cm}|p{5cm}}
    \textbf{parameter} &  \textbf{possible values} & \textbf{applicable to}  & \textbf{meaning}\\
    \hline
    \texttt{SHAPE}     &  ELLIPSE, RECTANGLE, DIAMOND & vertices& ellipse,
                                                                  rectangle, diamond\\
                       &  SQUARE, CYCLE & vertices & not implemented yet\\
    \hline
    \texttt{STROKE} or \texttt{STROKE COLOR} (synonyms)
                       &  WHITE, BLACK, BLUE, GREEN, RED, GRAY, \mbox{YELLOW},
                         CYAN, MAGENTA, SILVER, MAROON, OLIVE,
                         DARK\_GREEN, PURPLE, TEAL, NAVY, ORANGE
                                                   & both & the colour of the
                                                            border; other
                                                            colours are
                                                            possible as
                                                            properties of
                                                            vertices, but not
                                                            supported by the
                                                            filter
                                                            command\\
    \hline
    \texttt{EDGE TYPE} & SHARP, ROUND, BEZIER & edges & sharp or rounded
                                                        corners or a
                                                        B\'{e}zier curve\\
    \hline
    \texttt{FILL} or \texttt{FILL COLOR} (synonyms)
                       & as for \texttt{STROKE} & vertices & the fill color,
                                                             otherwise as for
                                                             \texttt{STROKE}\\
    \hline
    \texttt{WIDTH}     & float                 & vertices & \tcr{???}\\
    \hline
    \texttt{THICKNESS} & float                 & both\tcr{?} & \textcolor{red}{???}\\
    \hline
    \texttt{HEIGHT}     & float                & vertices  & \textcolor{red}{???}\\
    \hline
    \texttt{SIZE} & float                      & vertices \tcr{?}  & \textcolor{red}{???}\\
    \hline
    \texttt{WEIGHT} & float                      & edges  &
                                                            \textcolor{red}{???}\\
    \hline
    \texttt{ID}      & non-negative integer   & both & the internal unique id of
                                                       the vertex or of the edge\\
    \hline
    \texttt{DEGREE}    & non-negative integer & vertices & the degree of
                                                           the vertex;
                                                           only
                                                           undirected
                                                           edges count\\
    \hline
    \texttt{INDEGREE}    & non-negative integer & vertices & the in-degree of
                                                             the vertex;
                                                             only
                                                             directed
                                                             edges count\\
    \hline
    \texttt{OUTDEGREE}    & non-negative integer & vertices & the out-degree of
                                                              the vertex;
                                                              only
                                                              directed
                                                              edges count\\
    \hline
    \texttt{HAS SELFLOOP}  & no value & vertices & true if the vertex
                                                   has a self-loop\\
    \hline
    \texttt{NO SELFLOOP}  & no value & vertices & true if the vertex
                                                  has no self-loop\\
    \hline
    \texttt{LABEL EMPTY} or \texttt{NO LABEL} (synonyms)
                       & no value & vertices & true if the label is
                                               empty\\
    \hline
    \texttt{LABEL CONTAINS} & string & vertices & true if the label
                                                  contains the given string\\
    \hline
  \end{tabular}
  \caption{Parameters and their values}
  \label{tab:parsAndVals}
\end{table}

\subsection{Sorting the lists}

Before connecting lists of vertices (recall Example~\ref{ex:running})
we probably want to sort them, so that we know what vertex has
number~$i$. The lists can be sorted by

\begin{tcolorbox}[colback=lightpink]
  SORT <list id> <parameter>
\end{tcolorbox}%
where \texttt{parameter} is one of \texttt{LEFTRIGHT},
\texttt{RIGHTLEFT}, \texttt{TOPDOWN}, \texttt{BOTTOMUP}, \texttt{ID
  ASC}, \texttt{ID DECS}, \texttt{LABEL ASC}, \texttt{LABEL DESC}. The
former four parameters imply a sorting according to the corresponding
coordinate of vertices on the screen, \texttt{ID ASC} sorts by the id
in  the ascending order, \texttt{ID DESC} sorts by the id
in  the descending order and \texttt{LABEL} sorts alphabetically
(ascending or descending) by the label. If the labels of two vertices
are equal, the vertices keep their relative order.

\subsection{Operations on lists (before connecting)}

The following operations are currently supported:

\begin{tcolorbox}[colback=lightpink]
  \begin{tabular}[H]{p{7cm}|p{8cm}}
    \textbf{operation}    & \textbf{meaning}\\
    DELETE <list id> & deletes the list, but not the elements of the list from the graph\\

    UNION <source1> <source2> TO <target> & creates the union of \texttt{source1} and \texttt{source2} and stores the result in \texttt{target}\\

    INTERSECTION <list1 id> <list2 id> TO <list3 id> &creates the intersection of \texttt{source1} and \texttt{source2} and stores the result in \texttt{target}\\
    
    DIFFERENCE <source1> <source2> TO <target> & creates the difference of \texttt{source1} and \texttt{source2} and stores the result in \texttt{target}\\
    
    SYMMETRIC DIFFERENCE <source1> <source2> TO <target>  & creates the symmetric difference of \texttt{source1} and \texttt{source2} and stores the result in \texttt{target}\\
    
    COMPLEMENT <list id> TO <target>  & creates the complement of \texttt{list id} and stores the result in \texttt{target}.\\
    
    CONTRACT <vertex list id> | <edge list id> | SELECTED & builds a minor of the graph by contracting the subgraph induced by the elements of the list or by the selected elements (if \texttt{SELECTED} is specified and some elements are selected) to a single vertex. Works for all kinds of graphs. Not implemented yet.\\
    BUTTERFLY CONTRACT <edge list id> & builds a butterfly minor of the graph by butterfly contracting all butterfly contractable edges. On undirected edges the operation has no effect. Not implemented yet.\\

    SUBDIVIDE <edge list id> <$n$> & subdivides every edge from the list $n$ times where $n$ must be a natural number. Directed edges are replaced by directed paths of length $n+1$. Not implemented yet.
  \end{tabular}
\end{tcolorbox}

\subsection{Connecting lists of vertices}

One can either connect one list of vertices to another, which, however, can be the same list, or connect vertices of one list in a certain way. The first is done by the command

\begin{tcolorbox}[colback=lightpink]
  CONNECT <list1> <list2> =<formula> | BICLIQUE | MATCHING
\end{tcolorbox}

The easier case is to use a standard way of connection, for example,

\begin{tcolorbox}[colback=palegreen]
  CONNECT upper lower MATCHING
\end{tcolorbox}
which produces a matching connecting the $i$th vertex of
\texttt{list1} to the $i$th vertex of \texttt{list2}

%%% Local Variables:
%%% mode: latex
%%% TeX-master: "main"
%%% End:




\end{document}


%%% Local Variables:
%%% mode: latex
%%% TeX-master: t
%%% End:
